% !TeX root = informe.tex
\usepackage{alphalph}
\makeatletter
\newcommand*{\fnsymbolsingle}[1]{%
	\ensuremath{%
		\ifcase#1%
		\or *%
		\or \dagger
		\or \ddagger
		\or \mathsection
		\or \mathparagraph
 		\or	\diamond
 		\or	\aleph
% 		\or	\backepsilon %needs amssymb or something
 		\or	\flat
		\else
		\@ctrerr
		\fi
	}%
}
 \newcommand*{\myfnsymbol}[1]{%
 	\myfnsymbolsingle{\value{#1}}%
 }

\makeatother
\newalphalph{\fnsymbolmult}[mult]{\fnsymbolsingle}{}
\renewcommand*{\thefootnote}{%
\fnsymbolmult{\value{footnote}}%
}









%\usepackage{alphalph}
%\makeatletter
% \newcommand*{\myfnsymbolsingle}[1]{%
% 	\ensuremath{%
% 		\ifcase#1% 0
% 		\or % 1
% 		*%   
% 		\or % 2
% 		\dagger
% 		\or % 3  
% 		\ddagger
% 		\or % 4   
% 		\mathsection
% 		\or % 5
% 		\mathparagraph
% 		\or
% 		\diamond
% 		\or
% 		\aleph
% 		\or
% 		\backepsilon
% 		\or
% 		\flat
% 		\else % >= 7
% 		\@ctrerr  
% 		\fi
% 	}%   
% }   
% \makeatother
% 
% \newcommand*{\myfnsymbol}[1]{%
% 	\myfnsymbolsingle{\value{#1}}%
% }
%%  remove upper boundary by multiplying the symbols if needed
%
% \newalphalph{\myfnsymbolmult}[mult]{\myfnsymbolsingle}{}
% \renewcommand*{\thefootnote}{%
% 	\myfnsymbolmult{\value{footnote}}%
% }