%% DIFFERENTIAL OPERATOR
\makeatletter
\providecommand*{\diff}%
{\@ifnextchar^{\DIfF}{\DIfF^{}}}
\def\DIfF^#1{%
	\mathop{\mathrm{\mathstrut d}}%
	\nolimits^{#1}\gobblespace}
\def\gobblespace{%
	\futurelet\diffarg\opspace}
\def\opspace{%
	\let\DiffSpace\!%
	\ifx\diffarg(%
	\let\DiffSpace\relax
	\else
	\ifx\diffarg[%
	\let\DiffSpace\relax
	\else
	\ifx\diffarg\{%
	\let\DiffSpace\relax
	\fi\fi\fi\DiffSpace}
	

\def\Matlab{\(\textrm{\textsc{Matlab}}\)}%^\textrm{®}

\def\metano{\ensuremath{\mathrm{CH}_4}}
\newcommand{\oxygen}{\ensuremath{\textrm{O}_2}}
\newcommand{\dioxcarb}{\ensuremath{\textrm{CO}_2}}
\newcommand{\dioxsulf}{\ensuremath{\textrm{SO}_2}}
\renewcommand{\d}[1]{\ensuremath{\operatorname{d}\!{#1}}}
\newcommand{\sat}{{\tiny\textrm{sat}}}
\newcommand{\SOx}{\ensuremath{\textrm{SO}_x}}
\newcommand{\NOx}{\ensuremath{\textrm{NO}_x}}

\def\micro{\ensuremath{\mu}}
\def\px{\ensuremath{\mathrm{px}}}
\def\pixrad{\ensuremath{I_{\px}}}
\def\radiance{\ensuremath{I_R}}
\def\radianceunits{\ensuremath{\text{W}\,\text{m}^{-2}\,\text{sr}^{-1}}}
\def\pixradunits{\ensuremath{\text{W}\,\text{m}^{-2}\,\text{sr}^{-1}\,\micro \text{m}^{-1} }}

\def\LEO{\ensuremath{{\mathrm{\tiny LEO}}}}
\def\earth{\ensuremath{\mathrm{tierra}}}
\def\sensor{\ensuremath{{\! \mathrm{\footnotesize sens}}}}



\def\simulationGraphic{
\begin{figure}[htb!]
\centering
\pgfplotsset{colormap/jet}
	\begin{tikzpicture}	
		\begin{axis}[view={30}{40}, width=0.7\textwidth,y dir=reverse,
		title={Radiancia de píxel a 100km de altura},xlabel={Longitud de onda [\micro m]},
		ylabel={Concentración de \metano [ppm]}, zlabel={\pixrad~  [\pixradunits]}]
		\addplot3 [surf, mesh/rows=7, shader=faceted interp]
		table[x=wl, y=ch, z=ir, col sep=comma] {plots/ch4ppmSurf-M7.csv};
		\addplot3 [mesh, black, mesh/rows=7, shader=faceted interp]		table[x=wl, y=ch, z=ir, col sep=comma] {plots/ch4ppmSurf-M7.csv};
		\end{axis}
	\end{tikzpicture}
	\caption{Grafico  de radiancia de píxel en función de longitud de onda y concentración de \metano.}
	\label{fig:ch4IrrVsPpm}
\end{figure}
}