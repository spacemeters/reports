\subsection{Curvas de absorción del metano}

A través de la librería mencionada anteriormente se realizó un análisis del espectro de gases en la atmósfera, en donde se hizo particular énfasis en la detección del metano y la influencia de los otros gases a diferentes longitudes de onda, realizado en el siguiente \href{https://drive.google.com/open?id=1dH4wrmwYlQlsIZE5aXH2mcc7xXUcFgpj}{código}. Los resultados (figuras) están en el anexo de figuras (sección \ref{subsec:figuras})

Primero, se analizó la transferencia de metano y se ve que presenta picos de absorción en 1.6, 2,4 y 3.3 µm como se muestra en la Figura \ref{fig:absorbMetano}, donde se puede observar que el valor de intensidad en el pico de absorción se incrementa a medida que aumenta el rango de longitud de onda.

Analizando solamente el pico de los 1.6 µm como se muestra en la Figura \ref{fig:analisis16}, se puede observar que la influencia de los otros gases no es significante, pero tiene la característica que el pico de absorción es el más chico de los tres.

El segundo caso que analizamos fue el del rango de los 2.4 µm. Como se muestra en la Figura \ref{fig:analisis24}, se puede observar que el pico de absorción es mayor que el caso anterior pero la influencia del agua es significativa.


Por último, para el caso del rango de los 3.3 µm, como se muestra en la Figura \ref{fig:analisis33} el pico de absorción es el mayor de todos para el metano, pero si analizamos la influencia del resto de los gases se puede ver que es muy significativa.

Como conclusión, se puede decir que de las tres longitudes de onda estudiadas, la que tiene menor influencia de los otros gases es la de 1.6 µm, pero también se debe analizar si el pico de absorción es lo suficientemente grande para poder ser detectado y, además, se tiene que considerar la influencia del piso de ruido que tengamos en la medición.