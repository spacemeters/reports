\def\headingtype{\bf \small}
\def\openspacelogo{\href{https://www.spaceisopen.com/}{\includegraphics[height=25mm, clip]{caratula/logo}}}

\loadgeometry{titlepage}
\begin{titlepage}
	\centering
	\begin{tikzpicture}[remember picture, overlay]
	\coordinate (top_right) at 
	([xshift=-2.5cm, yshift=-2.5cm]current page.north east);
	\coordinate (top_left) at 
	([xshift=2.3cm, yshift=-2.3cm]current page.north west);
	\coordinate (bottom_right) at 
	([xshift=-1.8cm, yshift=1.8cm]current page.south east);
	\node[inner sep=0, anchor=north west] at (top_left) {};
	\node[yshift = 0.3cm, inner sep=0, anchor=north east] at (top_right) {
	\begin{tabular}{r}
	 {\headingtype \emph{}} \\ [-2pt]
		{\headingtype \carrera} \\[-2pt]
		{\headingtype \emph{\empresa}} \\[-2pt]	
	\end{tabular}
	};
	\draw[double, line width = 0.5pt, color = \colorborde] (top_left) rectangle (bottom_right);
	\end{tikzpicture}\par
	\vfill
	{\centering
		\includegraphics[width=4cm]{fig/SM.png}\par
	}
	{\Huge \bf  \tema \par}
	\vspace{2.0cm}
	{\LARGE \bf \titulo \par}
	\vspace{2.0cm}
	\begin{tabular}{c}
	\autor~
	\end{tabular}\par
	
	\vspace{1cm}
	
\begin{tabular}{c}
    %  \textbf{Resumen} \\ [10pt]
\end{tabular}\par
\begin{changemargin}{2cm}{2cm} % este comando es personalizado. ver preambulo
{ \resumen\par }
\end{changemargin}\par

	\vspace{2.0cm}
	{\Large \fecha \par}
	\vfill
	\openspacelogo{}
\end{titlepage}
\loadgeometry{main}

 %El siguiente documento tiene la finalidad de plasmar el grado de avance del proyecto haciendo un enfoque en particular en la división de las tareas para llevar a cabo en la etapa de factitibilidad y en detalle de los aspectos más relevantes para el desarrollo del proyecto, en función de esto organizamos equipos de trabajo, para abordar cada una de las tareas.